\documentclass[%
reprint,
superscriptaddress,
groupedaddress,
superscriptaddress,
onecolumn,
10pt
]{revtex4-2}
\usepackage[margin=2cm]{geometry}
\usepackage{graphicx,braket,bm,hyperref,amsmath,amssymb}

\begin{document}

\title{Quantum criticality in a Kondo-Mott lattice model}

\author{Abhirup Mukherjee}
\author{Siddhartha Lal}

\date{\today}

\maketitle
\section{Impurity Model}
\begin{equation}\begin{aligned}
	H_\text{aux}({\bf r}_d) = H^{(0)} + H_f({\bf r}_d) + H_c({\bf r}_d) + H_{fc}({\bf r}_d)~,
\end{aligned}\end{equation}
\begin{equation}\begin{aligned}
	H^{(0)} =& -t_f\sum_{\braket{i,j},\sigma}\left(f^\dagger_{i,\sigma}f_{j,\sigma} + \text{h.c.}\right) -t\sum_{\braket{i,j},\sigma}\left(c^\dagger_{i,\sigma}c_{j,\sigma} + \text{h.c.}\right) - \mu\sum_{i,\sigma}\left(f^\dagger_{i,\sigma}f_{i,\sigma} + c^\dagger_{i,\sigma}c_{i,\sigma}\right) ~,\\
	H_f({\bf r}_d) =& V_f\sum_{Z \in \text{NN}}\sum_{\sigma}\left(f^\dagger_{{\bf r}_d,\sigma} f_{Z,\sigma} + \text{h.c.}\right) + \epsilon_f \sum_\sigma f^\dagger_{{\bf r}_d,\sigma}f_{{\bf r}_d,\sigma} + U_f f^\dagger_{{\bf r}_d,\uparrow}f_{{\bf r}_d,\uparrow} f^\dagger_{{\bf r}_d,\downarrow}f_{{\bf r}_d,\downarrow} \\
				   &+ J_f\sum_{Z \in \text{NN}}\sum_{\alpha,\beta}{\bf S}_f({\bf r}_d) \cdot {\boldsymbol{\sigma}}_{\alpha\beta}f^\dagger_{Z,\alpha}f_{Z,\beta} - \frac{W_f}{2} \sum_{Z \in \text{NN}}\left(f^\dagger_{Z,\uparrow}f_{Z,\uparrow} - f^\dagger_{Z,\downarrow}f_{Z,\downarrow} \right)^2~,\\
	H_c({\bf r}_d) =& - \frac{W}{2}\left(c^\dagger_{{\bf r}_d,\uparrow}c_{{\bf r}_d,\uparrow} - c^\dagger_{{\bf r}_d,\downarrow}c_{{\bf r}_d,\downarrow} \right)^2~,\\
	H_{fc}({\bf r}_d) =& J\sum_{\alpha,\beta}{\bf S}_f({\bf r}_d) \cdot {\boldsymbol{\sigma}}_{\alpha\beta}c^\dagger_{{\bf r}_d,\alpha}c_{{\bf r}_d,\beta} + V \left(f^\dagger_{{\bf r}_d,\sigma} c_{{\bf r}_d,\sigma} + \text{h.c.}\right)~,\\
\end{aligned}\end{equation}

\section{Tiling reconstruction}
\begin{equation}\begin{aligned}
	H_\text{tiled} =& \sum_{{\bf r}_d}H_\text{aux}({\bf r}_d) - (N-1) H^{(0)} \\
	=& \sum_{\braket{i,j},\sigma}\left[-t_f\left(f^\dagger_{i,\sigma}f_{j,\sigma} + \text{h.c.}\right) -t\left(c^\dagger_{i,\sigma}c_{j,\sigma} + \text{h.c.}\right)\right] + \tilde J \sum_{\braket{i,j}}{\bf S}_f(i) \cdot {\bf S_f}(j) + J\sum_i {\bf S}_f(i)\cdot {\bf S}_c(i) - U\sum_i \left(f^\dagger_{i,\uparrow}f_{i,\uparrow} - f^\dagger_{i,\downarrow}f_{i,\downarrow}\right)^2 \\
	 &- \mu N
\end{aligned}\end{equation}

\section{Coupling renormalisation group flows}
Off-diagonal terms:
\begin{equation}\begin{aligned}
	H_{X, f} &= \frac{1}{2}\sum_{{\bf q},{\bf k},\sigma}J_f({\bf k},{\bf q}) S_f^\sigma \left(f^\dagger_{{\bf q},-\sigma}f_{{\bf k},\sigma} + f^\dagger_{{\bf k},-\sigma}f_{{\bf q},\sigma}\right) + \frac{1}{2}\sum_{{\bf q},{\bf k},\sigma}J_f({\bf k},{\bf q}) \sigma S_f^z \left(f^\dagger_{{\bf q},\sigma}f_{{\bf k},\sigma} + f^\dagger_{{\bf k},\sigma}f_{{\bf q},\sigma}\right)~,\\
	H_{X, c} &= \frac{1}{2}J\sum_{{\bf q},{\bf k},\sigma} S_f^\sigma \left(c^\dagger_{{\bf q},-\sigma}c_{{\bf k},\sigma} + c^\dagger_{{\bf k},-\sigma}c_{{\bf q},\sigma}\right) + \frac{1}{2}J\sum_{{\bf q},{\bf k},\sigma} \sigma S_f^z \left(c^\dagger_{{\bf q},\sigma}c_{{\bf k},\sigma} + c^\dagger_{{\bf k},\sigma}c_{{\bf q},\sigma}\right)~.
\end{aligned}\end{equation}

\subsection{Intra-layer processes}
\begin{equation}\begin{aligned}
	\Delta J_f^{(j)}\left({\bf k}_1, {\bf k}_2\right) &= -\sum_{{\bf q} \in \text{PS}} \left(J_f^{(j)}\left({\bf k}_2,{\bf q}\right) J_f^{(j)}\left({\bf q},{\bf k}_1\right) + 4J_f^{(j)}\left({\bf q}, {\boldsymbol \pi} + {\bf q}\right) W_{{\boldsymbol \pi} + {\bf q}, {\bf k}_2, {\bf k}_1, {\bf q}}\right) G_f(\omega, {\bf q})~,\\
	\Delta J^{(j)} &= - \rho(\varepsilon_j)\Delta \varepsilon\cdot\left[\left(J^{(j)}\right)^2 + 4WJ^{(j)}\right]~G(\omega, {\bf q})~,
\end{aligned}\end{equation}
where \({\bf \bar q} = {\boldsymbol \pi} + {\bf q}\) is the charge conjugate partner of \({\bf q}\), and the propagators \(G_f\) and \(G\) are defined as
\begin{equation}\begin{aligned}\label{propagators}
	G_f(\omega,{\bf q}) &= \frac{1}{2}\left[\left(\omega - \frac{1}{2}\left(|\varepsilon_f({\bf q})| - \mu\right) + J^{(j)}_f\left({\bf q},{\bf q}\right)/4 + W_f({\bf q})/2 - \epsilon_f\right)^{-1} + \left(\omega - \frac{1}{2}\left(|\varepsilon_f({\bf q})| + \mu\right) + J^{(j)}_f\left({\bf q},{\bf q}\right)/4 + W_f({\bf q})/2 - \epsilon_f\right)^{-1}\right] ~,\\
	G(\omega,{\bf q}) &= \frac{1}{2}\left[\left(\omega - \frac{1}{2}|\varepsilon({\bf q})| + J^{(j)}/4 + W/2 + \mu/2\right)^{-1} + \left(\omega - \frac{1}{2}|\varepsilon({\bf q})| + J^{(j)}/4 + W/2 - \mu/2\right)^{-1}\right] ~.
\end{aligned}\end{equation}


\subsection{Inter-layer processes}
Processes that start from configurations in which \({\bf q}\) is occupied:
\begin{equation}\begin{aligned}
	P_1 = \frac{1}{8}\sum_{{\bf q},{\bf k},{\bf k}_1, {\bf k}_2,\sigma,\sigma^\prime} J_f^{(j)}({\bf q},{\bf k}) S_f^\sigma f^\dagger_{{\bf q},-\sigma}f_{{\bf k},\sigma} ~ G_f(\omega, {\bf q}, {\bf k}) ~ J^{(j)} S_f^z \sigma^\prime c^\dagger_{{\bf k}_1,\sigma^\prime}c_{{\bf k}_2,\sigma^\prime} ~ G_f(\omega, {\bf q}, {\bf k}) ~ J_f^{(j)}({\bf q},{\bf k}) S_f^{-\sigma} f^\dagger_{{\bf k},\sigma}f_{{\bf q},-\sigma}~,
\end{aligned}\end{equation}
where the propagator \(G_f(\omega, {\bf q}, {\bf k})\) for the excitations is a generalisation of eq.~\ref{propagators}:
\begin{equation}\begin{aligned}
	G_f(\omega, {\bf q}, {\bf k}) &= \left(\omega - \frac{1}{2}|\varepsilon_f({\bf q})| - \frac{1}{2}|\varepsilon_f({\bf k})| + J^{(j)}_f\left({\bf q},{\bf q}\right)/4 + J^{(j)}_f\left({\bf k},{\bf k}\right)/4 + W_f({\bf q})/2 + W_f({\bf k})/2 - \epsilon_f\right)^{-1} ~,\\
\end{aligned}\end{equation}

Using \(S^\sigma S^z = -\frac{\sigma}{2}S^\sigma\) and \(S^\sigma S^{-\sigma} = \frac{1}{2} + \sigma S^z\), we get
\begin{equation}\begin{aligned}
	P_1 &= \sum_{{\bf k}_1, {\bf k}_2,\sigma,\sigma^\prime} \frac{-\sigma^\prime}{16} \left(\frac{\sigma}{2} + S_f^z\right) c^\dagger_{{\bf k}_1,\sigma^\prime}c_{{\bf k}_2,\sigma^\prime} ~  J^{(j)} \sum_{{\bf q} \in \text{PS} ,{\bf k} \in \text{HS}}\left(J_f^{(j)}({\bf q},{\bf k})\right)^2 G^2_f(\omega, {\bf q}, {\bf k})\\
		&= -\frac{1}{8}\sum_{{\bf k}_1, {\bf k}_2,\sigma^\prime} \sigma^\prime~S_f^z c^\dagger_{{\bf k}_1,\sigma^\prime}c_{{\bf k}_2,\sigma^\prime} ~  J^{(j)} \sum_{{\bf q} \in \text{PS} ,{\bf k} \in \text{HS}}\left(J_f^{(j)}({\bf q},{\bf k})\right)^2 G^2_f(\omega, {\bf q}, {\bf k})~.
\end{aligned}\end{equation}

Another process can be conceived with similar starting configuration but where the loop momenta are on the \(c-\)plane:
\begin{equation}\begin{aligned}
	P_2 = \frac{1}{8}\sum_{{\bf q},{\bf k},{\bf k}_1, {\bf k}_2,\sigma,\sigma^\prime} J^{(j)} S_f^\sigma c^\dagger_{{\bf q},-\sigma}c_{{\bf k},\sigma} ~ \tilde \tilde G(\omega, {\bf q}, {\bf k}) ~ J^{(j)}({\bf k}_1,{\bf k}_2) S_f^z \sigma^\prime f^\dagger_{{\bf k}_1,\sigma^\prime}f_{{\bf k}_2,\sigma^\prime} ~ \tilde G(\omega, {\bf q}, {\bf k}) ~ J^{(j)} S_f^{-\sigma} c^\dagger_{{\bf k},\sigma}c_{{\bf q},-\sigma}~,
\end{aligned}\end{equation}
where \(\tilde G(\omega,{\bf q}, {\bf k})\) is defined as
\begin{equation}\begin{aligned}
	\tilde G(\omega,{\bf q}, {\bf k}) &= \frac{1}{\omega - \frac{1}{2}|\varepsilon({\bf q})| - \frac{1}{2}|\varepsilon({\bf k})| + J^{(j)}/2 + W}~.
\end{aligned}\end{equation}


Using the same properties as above, the expression simplifies to:
\begin{equation}\begin{aligned}
	P_2 &= -\frac{1}{8}\sum_{{\bf k}_1, {\bf k}_2,\sigma^\prime} \sigma^\prime~S_f^z f^\dagger_{{\bf k}_1,\sigma^\prime}f_{{\bf k}_2,\sigma^\prime} ~  J_f^{(j)}({\bf k}_1,{\bf k}_2) \left(J^{(j)}\right)^2 \sum_{{\bf q} \in \text{HS} ,{\bf k} \in \text{PS}}\tilde G^2(\omega,{\bf q}, {\bf k})~.
\end{aligned}\end{equation}

It turns out that the processes that start from unoccupied configurations in the state \({\bf q}\) give almost identical contributions as the present expressions, the only difference being that \({\bf q}\) is now summed over the hole sector (HS) while \({\bf k}\) is summed over the particle sector (PS). For a particle-hole symmetric system (which we have restricted ourselves to), these two summations are equal, so both the contributions are indeed identical. The total renormalisation can therefore be read off from the final expressions of \(P_1\) and \(P_2\) (and the hole sector is accounted for by doubling the renormalisation from just the particle sector):
\begin{equation}\begin{aligned}
	\Delta J_f^{(j)}({\bf k}_1,{\bf k}_2) = -\frac{1}{2}J_f^{(j)}({\bf k}_1,{\bf k}_2) \left(J^{(j)}\right)^2 \sum_{{\bf q} \in \text{PS} ,{\bf k} \in \text{HS}}\tilde G^2(\omega, {\bf q}, {\bf k})\\
	\Delta J^{(j)} = -\frac{1}{2}J^{(j)} \sum_{{\bf q} \in \text{PS} ,{\bf k} \in \text{HS}}\left(J_f^{(j)}({\bf q},{\bf k}) G_f(\omega, {\bf q}, {\bf k})\right)^2\\
\end{aligned}\end{equation}

\subsection{Complete coupling RG equation}
\begin{equation}\begin{aligned}\label{rgEquation}
	\Delta J_f^{(j)}\left({\bf k}_1, {\bf k}_2\right) &= -\sum_{{\bf q} \in \text{PS}} \left[\left(J_f^{(j)}\left({\bf k}_2,{\bf q}\right) J_f^{(j)}\left({\bf q},{\bf k}_1\right) + 4J_f^{(j)}\left({\bf q}, {\bf \bar q}\right) W_{{\bf \bar q}, {\bf k}_2, {\bf k}_1, {\bf q}}\right) G_f(\omega, {\bf q}) + \frac{1}{2}J_f^{(j)}({\bf k}_1,{\bf k}_2) \left(J^{(j)}\right)^2 \sum_{{\bf k} \in \text{HS}}\tilde G^2(\omega, {\bf q}, {\bf k})\right]~,\\
	\Delta J^{(j)} &= - \rho(\varepsilon_j)\Delta \varepsilon\cdot\left[\left(J^{(j)}\right)^2 + 4WJ^{(j)}\right] G(\omega, {\bf q}) -\frac{1}{2}J^{(j)} \sum_{{\bf q} \in \text{PS} ,{\bf k} \in \text{HS}}\left(J_f^{(j)}({\bf q},{\bf k})G_f(\omega, {\bf q}, {\bf k})\right)^2~,
\end{aligned}\end{equation}

\subsection{Symmetries preserved under renormalisation}
By Fourier transforming the real-space forms of the Kondo coupling \(J_f\) and bath interaction \(W\), we get their \(k-\)space forms:
\begin{equation}\begin{aligned}\label{momentumForm}
	J_f({\bf k}, {\bf q}) &= \frac{J_f}{2}\left[\cos(k_x - q_x) + \cos(k_y - q_y)\right]~,\\
	W({\bf k}, {\bf q}, {\bf k}^\prime, {\bf q}^\prime) &= \frac{W}{2}\left[\cos(k_x - q_x + k^\prime_x - q^\prime_x) + \cos(k_y - q_y + k^\prime_y - q^\prime_y)\right]~.
\end{aligned}\end{equation}
These are of course the unrenormalised forms; the Kondo coupling \(k-\)space dependence can evolve during the RG flow. The \(k-\)space sensitive form of the Kondo coupling and conduction bath interactions are invariant under symmetry transformations in the Brillouin zone.\\

\par\noindent{\bf Translation by a nesting vector into opposite quadrant}\\
Define the reciprocal lattice vectors (RLVs) \({\bf Q}_1 = \left(\pi, \pi\right)\) and \({\bf Q}_2 = \left(\pi, -\pi\right)\). The bare Kondo coupling in eq.~\ref{momentumForm} is (anti)symmetric under translation of (one)both momentum by either of the two RLVs:
\begin{equation}\begin{aligned}\label{oppositeQ}
	J_f({\bf k} + {\bf Q}_i, {\bf q}) = J_f({\bf k}, {\bf q} + {\bf Q}_i) = - J_f({\bf k}, {\bf q})~;~i = 1,2~;\\
	J_f({\bf k} + {\bf Q}_i, {\bf q} + {\bf Q}_j) = J_f({\bf k}, {\bf q})~;~i=1,2~~j=1,2~.\\
\end{aligned}\end{equation}
These symmetries survive under the renormalisation group transformations. For the first transformation (under which the Kondo coupling is antisymmetric), note that each of the terms on the right hand side of the RG equation for \(J_f\) (eq.~\ref{rgEquation}) are antisymmetric as well - the third term because it's the Kondo coupling itself which automatically has the symmetry, the second term (involving \(W\)) because \(W\) is also antisymmetric under transformation of one momentum, and the first term because only one of the two \(J_f\) in the product will transform (to obtain a minus sign). This ensures that the entirety of the renormalisation transforms antisymmetrically. A very similar argument shows that the symmetry under transformation of both momenta also survives under renormalisation.\\

\par\noindent{\bf Translation into adjacent quadrant}\\
We will make use of another symmetry. Consider two momenta \({\bf k}\) and \({\bf q}\) in the first and second quadrant, with the same \(y-\)component but opposite \(x-\)components:
\begin{equation}\begin{aligned}
	{\bf k}_y = {\bf q}_y, ~ {\bf k}_x = -{\bf q}_x~.
\end{aligned}\end{equation}
We refer to \({\bf q}\) as \({\bf \bar k}\) to signal the fact the above relation between the two momenta. We first consider the bare interaction, where we have the symmetry
\begin{equation}\begin{aligned}\label{adjacent}
	J_f({\bf k}, {\bf \bar k}^\prime) = J_f({\bf \bar k}, {\bf k}^\prime),~\\
	J_f({\bf \bar k}, {\bf \bar k}^\prime) = J_f({\bf k}, {\bf k}^\prime)~.
\end{aligned}\end{equation}
We now argue that these symmetries are preserved during the RG flow. Using the properties \(W_{{\bf \bar q}, {\bf k}_2, {\bf \bar k}_1, {\bf q}} = W_{{\bf \bar q}, {\bf \bar k}_2, {\bf k}_1, {\bf q}}\) and  \(J_f^{(j)}\left({\bf k}_2,{\bf q}\right) J_f^{(j)}\left({\bf q},{\bf \bar k}_1\right) = J_f^{(j)}\left({\bf \bar k}_2,{\bf \bar q}\right) J_f^{(j)}\left({\bf \bar q},{\bf k}_1\right)\) and the fact that \({\bf \bar q}\) lies on the same isoenergy shell as \({\bf q}\) and is already part of the summation over PS in eq.~\ref{rgEquation}, we can see that \(\Delta J_f^{(j)}\left({\bf \bar k}_1, {\bf k}_2\right) = \Delta J_f^{(j)}\left({\bf k}_1, {\bf \bar k}_2\right)\). A similar line of argument shows that \(\Delta J_f^{(j)}\left({\bf \bar k}_1, {\bf \bar k}_2\right) = \Delta J_f^{(j)}\left({\bf k}_1, {\bf k}_2\right)\).


\section{Reduction to Truncated 1D Representation}
At the renormalisation group fixed point, we have a renormalised theory for the interaction of the impurity spin \(S_f\) with the \(f-\)layer Fermi surface:
\begin{equation}\begin{aligned}
	H^* = \sum_{\alpha,\beta}{\bf S}_f\cdot{\boldsymbol\sigma}_{\alpha,\beta}\sum_{{\bf k}, {\bf q}} J_f^*({\bf k},{\bf q}) f^\dagger_{{\bf k},\alpha}f_{{\bf q},\beta}~.
\end{aligned}\end{equation}
We will now obtain a more minimal representation of this interaction. Each momentum label is summed over all four quadrants \(\mathcal{Q}_1\) through \(\mathcal{Q}_4\); eq.~\ref{oppositeQ} relates \(\mathcal{Q}_1\) with \(\mathcal{Q}_3\) and \(\mathcal{Q}_2\) with \(\mathcal{Q}_4\):
\begin{equation}\begin{aligned}
	\sum_{{\bf k}, {\bf q}} J_f^*({\bf k},{\bf q}) f^\dagger_{{\bf k},\alpha}f_{{\bf q},\beta} 
	&= \sum_{\bf q}\left[\sum_{{\bf k} \in \mathcal{Q}_1,\mathcal{Q}_2} J_f^*({\bf k},{\bf q}) f^\dagger_{{\bf k},\alpha} + \sum_{{\bf k} \in \mathcal{Q}_3,\mathcal{Q}_4} J_f^*({\bf k},{\bf q}) f^\dagger_{{\bf k},\alpha}\right] f_{{\bf q},\beta}\\
	&= \sum_{\bf q}\left[\sum_{{\bf k} \in \mathcal{Q}_1,\mathcal{Q}_2} J_f^*({\bf k},{\bf q}) f^\dagger_{{\bf k},\alpha} + \sum_{{\bf k} \in \mathcal{Q}_1} J_f^*({\bf k - Q_1},{\bf q}) f^\dagger_{{\bf k - Q_1},\alpha} + \sum_{{\bf k} \in \mathcal{Q}_2} J_f^*({\bf k + Q_2},{\bf q}) f^\dagger_{{\bf k + Q_2},\alpha}\right] f_{{\bf q},\beta}\\
	&= \sum_{\bf q}\left[\sum_{{\bf k} \in \mathcal{Q}_1} J_f^*({\bf k},{\bf q}) \left(f^\dagger_{{\bf k},\alpha} - f^\dagger_{{\bf k - Q_1},\alpha}\right) + \sum_{{\bf k} \in \mathcal{Q}_2} J_f^*({\bf k},{\bf q}) \left(f^\dagger_{{\bf k},\alpha} - f^\dagger_{{\bf k + Q_2},\alpha}\right) \right] f_{{\bf q},\beta}~.
\end{aligned}\end{equation}
For ease of notation, we define new fermionic operators \(A_{{\bf k},\sigma}\) and \(B_{{\bf k},\sigma}\):
\begin{equation}\begin{aligned}\label{defineA}
	A_{{\bf k},\sigma,\pm} = \frac{1}{\sqrt{2}}\left(f_{{\bf k},\sigma} \pm f_{{\bf k - Q_1},\sigma}\right), {\bf k}\in\mathcal{Q}_1~,\\
	B_{{\bf k},\sigma,\pm} = \frac{1}{\sqrt{2}}\left(f_{{\bf k},\sigma} \pm f_{{\bf k + Q_2},\sigma}\right), {\bf k}\in\mathcal{Q}_2~,
\end{aligned}\end{equation}
which satisfy the appropriate algebra: \(\left\{A_{{\bf k},\sigma, p}, A_{{\bf k}^\prime,\sigma^\prime, p^\prime}\right\} = \left\{B_{{\bf k},\sigma, p}, B_{{\bf k}^\prime,\sigma^\prime, p^\prime}\right\} = \delta_{{\bf k},{\bf k}^\prime}\delta_{\sigma,\sigma^\prime}\delta{p, p^\prime}\) and \(\left\{A_{{\bf k},\sigma,p}, B_{{\bf k}^\prime,\sigma^\prime,p^\prime}\right\} = 0\), with \(p=\pm\) denoting the flavours of the \(A\) and \(B\) fields. Note that only the \(p=-1\) flavour enters the Hamiltonian. Henceforth, we drop the label \(\pm\) and it is implied that \(A\) and \(B\) refer to the \(p=-1\) variants.

Decomposing the sum over \({\bf q}\) in a similar fashion, we get
\begin{equation}\begin{aligned}
	\sum_{{\bf k}, {\bf q}} J_f^*({\bf k},{\bf q}) f^\dagger_{{\bf k},\alpha}f_{{\bf q},\beta} 
	&= 2\sum_{{\bf k} \in \mathcal{Q}_1, {\bf q} \in \mathcal{Q}_1} J_f^*({\bf k},{\bf q}) A^\dagger_{{\bf k},\alpha}A_{{\bf q},\beta} + 2\sum_{{\bf k} \in \mathcal{Q}_2, {\bf q} \in \mathcal{Q}_1} J_f^*({\bf k},{\bf q}) B^\dagger_{{\bf k},\alpha}A_{{\bf q},\beta} \\
	&+ 2\sum_{{\bf k} \in \mathcal{Q}_1, {\bf q} \in \mathcal{Q}_2} J_f^*({\bf k},{\bf q}) A^\dagger_{{\bf k},\alpha}B_{{\bf q},\beta} + 2\sum_{{\bf k} \in \mathcal{Q}_2, {\bf q} \in \mathcal{Q}_2} J_f^*({\bf k},{\bf q}) B^\dagger_{{\bf k},\alpha}B_{{\bf q},\beta}~.
\end{aligned}\end{equation}

To further simplify things, We first replace the summations over \(\mathcal{Q}_2\) with that over \(\mathcal{Q}_1\), with the mapping \({\bf q} \to {\bf \bar q}\) and use eq.~\ref{adjacent}:
\begin{equation}\begin{aligned}
	\sum_{{\bf k}, {\bf q}} J_f^*({\bf k},{\bf q}) f^\dagger_{{\bf k},\alpha}f_{{\bf q},\beta} 
	&= 2\sum_{{\bf k} \in \mathcal{Q}_1, {\bf q} \in \mathcal{Q}_1} \left[J_f^*({\bf k},{\bf q}) A^\dagger_{{\bf k},\alpha}A_{{\bf q},\beta} + J_f^*({\bf \bar k},{\bf q}) B^\dagger_{{\bf k},\alpha}A_{{\bf \bar q},\beta} + J_f^*({\bf k},{\bf \bar q}) A^\dagger_{{\bf k},\alpha}B_{{\bf \bar q},\beta} + J_f^*({\bf \bar k},{\bf \bar q}) B^\dagger_{{\bf \bar k},\alpha}B_{{\bf \bar q},\beta}\right]\\
	&= 2\sum_{{\bf k} \in \mathcal{Q}_1, {\bf q} \in \mathcal{Q}_1} \left[J_f^*({\bf k},{\bf q}) \left(A^\dagger_{{\bf k},\alpha}A_{{\bf q},\beta} + B^\dagger_{{\bf \bar k},\alpha}B_{{\bf \bar q},\beta}\right) + J_f^*({\bf \bar k},{\bf q}) \left(B^\dagger_{{\bf k},\alpha}A_{{\bf \bar q},\beta} + A^\dagger_{{\bf k},\alpha}B_{{\bf \bar q},\beta}\right)\right] ~.
\end{aligned}\end{equation}
To remove all explicit references to operators in \(\mathcal{Q}_2\), we define a new set of operators:
\begin{equation}\begin{aligned}\label{defineGamma}
	\gamma_{{\bf k},\sigma,\pm} = \frac{1}{\sqrt{2}}\left(A_{{\bf k},\sigma,-} \pm B_{{\bf \bar k},\sigma,-}\right), {\bf k}\in \mathcal{Q}_1~,\\
	\phi_{{\bf k},\sigma,\pm} = \frac{1}{\sqrt{2}}\left(A_{{\bf k},\sigma,+} \pm B_{{\bf \bar k},\sigma,+}\right), {\bf k}\in \mathcal{Q}_1~,
\end{aligned}\end{equation}
where we have restored the \(p\)-values into the \(A\) and \(B\) fields in order to define two new fermionic fields: \(\left\{\gamma_{{\bf k},\sigma,\pm}, \gamma_{{\bf k}^\prime,\sigma^\prime,\pm}^\dagger\right\} = \delta_{{\bf k},{\bf k}^\prime}\delta_{\sigma,\sigma^\prime}\), \(\left\{\gamma_{{\bf k},\sigma,\pm}, \gamma_{{\bf k}^\prime,\sigma^\prime,\mp}^\dagger\right\} = 0\). In terms of these new fields, we finally obtain a Hamiltonian which is defined purely in the first quadrant \(\mathcal{Q}_1\) (this is however mostly formal because there are now twice as many modes on \(\mathcal{Q}_1\) than before):
\begin{equation}\begin{aligned}
	\sum_{{\bf k}, {\bf q}} J_f^*({\bf k},{\bf q}) f^\dagger_{{\bf k},\alpha}f_{{\bf q},\beta} 
	&= \sum_{{\bf k} \in \mathcal{Q}_1, {\bf q} \in \mathcal{Q}_1} \left[\left[J_f^*({\bf k},{\bf q}) + J_f^*({\bf \bar k},{\bf q})\right] \gamma^\dagger_{{\bf k},\alpha,+}\gamma_{{\bf q},\alpha,+} + \left[J_f^*({\bf k},{\bf q}) - J_f^*({\bf \bar k},{\bf q})\right] \gamma^\dagger_{{\bf k},\alpha,-}\gamma_{{\bf q},\alpha,-}\right] \\
\end{aligned}\end{equation}

\section{Correlations in truncated representation}
In order to calculate equal-time correlations (such as \(\braket{S_d^+ f^\dagger_{{\bf k}\downarrow}f_{{\bf k}^\prime \uparrow}} \)), we combine eqs.~\ref{defineA} and \ref{defineGamma} to express the bare fields \(f_{{\bf k},\sigma}\) in terms of the new fields \(\gamma_{{\bf k},\sigma,\pm}\) and \(\phi_{{\bf k},\sigma,\pm}\):
\begin{equation}\begin{aligned}
	&f_{{\bf k},\sigma} = \frac{1}{2}\left(\phi_{{\bf k},\sigma,+} + \phi_{{\bf k},\sigma,-} + \gamma_{{\bf k},\sigma,+} + \gamma_{{\bf k},\sigma,-}\right), ~ &&f_{{\bf \bar k},\sigma} = \frac{1}{2}\left(\phi_{{\bf k},\sigma,+} - \phi_{{\bf k},\sigma,-} + \gamma_{{\bf k},\sigma,+} - \gamma_{{\bf k},\sigma,-}\right),\\
	&f_{{\bf k} - {\bf Q}_1,\sigma} = \frac{1}{2}\left(\phi_{{\bf k},\sigma,+} + \phi_{{\bf k},\sigma,-} - \gamma_{{\bf k},\sigma,+} - \gamma_{{\bf k},\sigma,-}\right),~ &&f_{{\bf \bar k} + {\bf Q}_2,\sigma} = \frac{1}{2}\left(\phi_{{\bf k},\sigma,+} - \phi_{{\bf k},\sigma,-} - \gamma_{{\bf k},\sigma,+} + \gamma_{{\bf k},\sigma,-}\right),
\end{aligned}\end{equation}
where the four relations act on operators in four quadrants. Suppose that both \({\bf k}\) and \({\bf k}^\prime\) in the correlation \(\braket{S_d^+ f^\dagger_{{\bf k}\downarrow}f_{{\bf k}^\prime \uparrow}}\) are from the first quadrant. Since the fields \(\phi\) do not appear in the Hamiltonian, they will not contribute to the correlation measures in the absence of symmetry breaking and entanglement. As an example, the correlation defined above can be expressed as
\begin{equation}\begin{aligned}
	\braket{S_d^+ f^\dagger_{{\bf k}\downarrow}f_{{\bf k}^\prime \uparrow}} = \frac{1}{4}\braket{S_d^+ (\gamma^\dagger_{{\bf k},\downarrow,+} + \gamma^\dagger_{{\bf k},\downarrow,-})(\gamma_{{\bf k}^\prime,\uparrow,+} + \gamma_{{\bf k}^\prime,\uparrow,-})}~,
\end{aligned}\end{equation}
and so can be computed from the four correlations \(\braket{S_d^+ \gamma^\dagger_{{\bf k},\downarrow,\pm} \gamma_{{\bf k}^\prime,\uparrow,\pm}}\).

\section{Bilayer extended Hubbard model: Impurity model}
We approach the heavy-fermion problem by starting from a bilayer extended Hubbard model, consisting of two layers (\(f\) and \(c\)). Towards studying this lattice model, we adopt a two-layer impurity problem that hosts a correlated impurity site in each layer (\(S_f\) and \(S_d\)):
\begin{equation}\begin{aligned}
	H_\text{aux} = H_\text{iti} + H_f + H_d + H_{fd}~,
\end{aligned}\end{equation}
where \(H_\text{iti}\) is  the Hamiltonian for the non-interacting itinerant electrons of either layer,
\begin{equation}\begin{aligned}
H_\text{iti} = -\sum_{\sigma,\alpha}\left[t_\alpha\sum_{\braket{i,j}}\left(c^\dagger_{i,\sigma,\alpha}c_{j,\sigma,\alpha} + \text{h.c.}\right) + \mu\sum_{i,\sigma,\alpha}n_{i,\sigma,\alpha}\right] ~,
\end{aligned}\end{equation}
such that \(\alpha\) sums over the two layers \(f\) and \(d\). \(H_f\) and \(H_d\) describe the dynamics of the correlated impurity sites (and their local neighbourhood) in each layer:
\begin{equation}\begin{aligned}
	H_\alpha = \varepsilon_\alpha \sum_\sigma n_{\alpha,\sigma} + U_\alpha n_{\alpha,\uparrow} n_{\alpha,\downarrow} + \sum_{Z \in \text{NN}}\left[V_\alpha\sum_{\sigma}\left(\alpha^\dagger_{\sigma} c_{Z,\sigma,\alpha} + \text{h.c.}\right) + \frac{1}{2}J_\alpha\sum_{\alpha,\beta}{\bf S}_\alpha \cdot {\boldsymbol{\sigma}}_{\alpha\beta}c^\dagger_{Z,\alpha,\alpha}c_{Z,\beta,\alpha} - \frac{W_\alpha}{2} \left(n_{Z,\uparrow,\alpha} - n_{Z,\downarrow,\alpha}\right)^2\right] ~,\\
\end{aligned}\end{equation}
where \(\alpha^\dagger_{\sigma}\) can refer to creation operator for either the \(f-\)layer (\(f^\dagger_{\sigma}\)) or the \(d-\)layer (\(d^\dagger_{\sigma}\)). Finally, \(H_{fd}\) represents the inter-layer hybridisation:
\begin{equation}\begin{aligned}
	H_{fc} = J{\bf S}_f \cdot {\bf S}_d + V \sum_\sigma\left(f^\dagger_{\sigma} d_{\sigma} + \text{h.c.}\right)~,\\
\end{aligned}\end{equation}

Tiling the impurity model leads to bilayer extended Hubbard model. In order to tile, we place the impurity sites at a position \({\bf r}\) on the lattice, and then we translate the entire model, taking into account the overcounting of the itinerant electrons:
\begin{equation}\begin{aligned}
	H_\text{tiled} &= \sum_{{\bf r}}H_\text{aux}({\bf r}) - (N-1) H_\text{iti} \\
	&= \sum_\alpha\left[-\tilde t_\alpha\sum_{\braket{i,j},\sigma}\left(c^\dagger_{i,\sigma,\alpha}c_{j,\sigma,\alpha} + \text{h.c.}\right) + \tilde J \sum_{\braket{i,j}}{\bf S}_{i,\alpha} \cdot {\bf S}_{j,\alpha} + \varepsilon_\alpha \sum_{i,\sigma}n_{i,\sigma,\alpha} + U_\alpha \sum_i n_{i,\uparrow,\alpha}n_{i,\downarrow,\alpha} \right] \\
	 &+ \sum_i \left[J {\bf S}_{i,f} \cdot {\bf S}_{j,d} + V\sum_\sigma\left(c^\dagger_{i,\sigma,f}c_{i,\sigma,d} + \text{h.c.}\right) \right]	
\end{aligned}\end{equation}

\section{Unitary RG Analysis of bilayer lattice-embedded SIAM}
In the limit of large \(U_\alpha\), we carry out a Schrieffer-Wolff transformation and work with the following low-energy Hamiltonian:
\begin{equation}\begin{aligned}
	H_\text{aux} &= \sum_{{\bf k},\sigma,\alpha}\epsilon_{{\bf k},\alpha}n_{{\bf k},\sigma,\alpha} + \sum_\alpha\sum_{Z \in \text{NN}}\left[\frac{1}{2}J_\alpha\sum_{\sigma,\sigma^\prime}{\bf S}_\alpha \cdot {\boldsymbol{\sigma}}_{\alpha\beta}c^\dagger_{Z,\sigma,\alpha}c_{Z,\sigma^\prime,\alpha} - \frac{W_\alpha}{2} \left(n_{Z,\uparrow,\alpha} - n_{Z,\downarrow,\alpha}\right)^2\right] + J{\bf S}_f \cdot {\bf S}_d~.
\end{aligned}\end{equation}

In order to study the low-energy physics of the impurity model, we iteratively integrate out high-energy degrees of freedom using the unitary RG method. We already have the renormalisation group equations for the couplings \(J_\alpha\) in the case of \(J=0\). It turns out that even upon switching on \(J\), no additional processes exist which renormalise \(J_\alpha\). As a result, their RG equations remain unchanged.

We now turn to the renormalisation of \(J\), arising from hybridisation of \(f-\) and \(d-\)layers. Let the momentum states being decoupled from the two layers be \(\ket{{\bf q}_\pm,\sigma,f}\) and \(\ket{{\bf q}_\pm,\sigma,d}\), where the subscript indicates the sign of the energy of that state (and hence whether its occupancy in the low-energy configurations \(\ket{L}_f\), \(\ket{L}_d\)). In order to capture coherent scattering processes between the two layers, we project onto the following rotated basis of excited states:
\begin{equation}\begin{aligned}
	\ket{H}_\pm &= \frac{1}{\sqrt{2}}\left(\ket{H({\bf q})}_f \pm \ket{H({\bf q})}_d\right)~,\\
	\mathcal{P({\bf q})}_H &= \ket{H}_+\bra{H}_+ + \ket{H}_-\bra{H}_-~,
\end{aligned}\end{equation}
where \(\ket{H({\bf q})}_\alpha\) is the state obtained upon exciting both states \(\ket{\bf q}_\pm\) in the layer \(\alpha\): \(\ket{H(q)}_\alpha = c^\dagger_{\bf q_+,\alpha}c_{\bf q_-,\alpha}\ket{L}_f\ket{L}_d\), and \(\mathcal{P({\bf q})}_H\) projects onto this rotated excited basis. We focus on the spin-flip component \(J S_f^+ S_d^-\) of the Hamiltonian. There are two kinds of processes that renormalise this coupling. We first consider one that involves a spin-flip of the \(d-\)layer followed by a spin-flip of the \(f-\)layer:
\begin{equation}\begin{aligned}\label{rgEqtn1}
	\Delta H = \frac{1}{N^2}\sum_{{\bf q} \in \text{UV}}\braket{L | \frac{1}{2}J_f S_f^+ c^\dagger_{{\bf q-}_\downarrow,f}c_{{\bf q+}\uparrow,f} \mathcal{P({\bf q})}_H G \mathcal{P({\bf q})}_H\frac{1}{2}J_dS_d^- c^\dagger_{{\bf q+}\uparrow,d}c_{{\bf q-}\downarrow,d} | L }~,
\end{aligned}\end{equation}
where \(\ket{L} = \ket{L}_f \ket{L}_d\) is the initial configuration for both layers. \(G\) is the propagator for the excited state:
\begin{equation}\begin{aligned}
	G = \frac{1}{\omega - H_D}~,
\end{aligned}\end{equation}
where \(H_D\) is the diagonal part of the Hamiltonian corresponding to the excited(intermediate) state. The diagonal part consists of the kinetic energy and the Ising part of the Kondo interaction:
\begin{equation}\begin{aligned}
	H_D = \sum_{{\bf q,\sigma}\in{\bf q}_\pm, \alpha}\epsilon_\alpha({\bf q})\tau_{{\bf q},\sigma} + \sum_\alpha J_\alpha S_\alpha^z \sum_{{\bf q},\sigma}\sigma n_{{\bf q},\sigma,\alpha}~.
\end{aligned}\end{equation}
Calculating the matrix elements of the propagator in the rotated basis gives
\begin{equation}\begin{aligned}\label{PGP}
	\mathcal{P({\bf q})}_H G \mathcal{P({\bf q})}_H = \frac{1}{2}\left(G_f + G_d\right)\left(\ket{H}_+\bra{H}_+ + \ket{H}_-\bra{H}_-\right) + \frac{1}{2}\left(G_f - G_d\right)\left(\ket{H}_+\bra{H}_- + \ket{H}_-\bra{H}_+\right)~,
\end{aligned}\end{equation}
where
\begin{equation}\begin{aligned}
	G_\alpha = \frac{1}{\omega - \frac{1}{2}\left[\epsilon_\alpha(q_+) - \epsilon_\alpha(q_-)\right] + \frac{1}{2}J_\alpha + W_\alpha}~.
\end{aligned}\end{equation}

In eq.~\ref{rgEqtn1}, since the first process is an excitation into a \(d-\)state and the second process is a de-excitation from an \(f-\)state, the expression can be simplified into
\begin{equation}\begin{aligned}\label{rgEqtn2}
	\braket{L | S_f^+ c^\dagger_{{\bf q-}_\downarrow,f}c_{{\bf q+}\uparrow,f} \mathcal{P({\bf q})}_H G \mathcal{P({\bf q})}_H S_d^- c^\dagger_{{\bf q+}\uparrow,d}c_{{\bf q-}\downarrow,d} | L } = S_f^+ S_d^- \braket{H_f | \mathcal{P({\bf q})}_H G \mathcal{P({\bf q})}_H | H_d} = \frac{1}{2}\left(G_f + G_d\right) S_f^+ S_d^-~.
\end{aligned}\end{equation}
The net Hamiltonian renormalisation is finally a sum over the contribution from each mode:
\begin{equation}\begin{aligned}
	\Delta H = \frac{1}{8N^2}J_f J_d \sum_{{\bf q} \in \text{UV}}\left(G_f + G_d\right) S_f^+ S_d^-~.
\end{aligned}\end{equation}
Converting the sums to integrals (assuming a density of states \(\rho({\bf q})\)) gives the final expression
\begin{equation}\begin{aligned}
	\frac{\Delta H}{\Delta D} = \frac{1}{8N} S_f^+ S_d^-J_f J_d \int_{UV} d{\bf q} ~\rho({\bf q}) \left(G_f + G_d\right)~.
\end{aligned}\end{equation}

The time-reversed of this process can be obtained simply by exchanging the \(d-\)interaction with the \(f-\)interaction. Since the renormalisation itself is symmetric, the contribution is equal to what we obtained here. The total renormalisation in the coupling for the term \(S_f^+ S_d^-\) is therefore
\begin{equation}\begin{aligned}
	\frac{\Delta J}{\Delta D} &= \frac{1}{2}J_f J_d \int_{UV} d{\bf q} ~\rho({\bf q}) \left(G_f + G_d\right) \\
							  &= \frac{1}{2}J_f J_d \int_{UV} d{\bf q} ~\rho({\bf q}) \left[\frac{1}{\omega - \frac{1}{2}\left[\epsilon_f(q_+) - \epsilon_f(q_-)\right] + \frac{1}{2}J_f + W_f} + \frac{1}{\omega - \frac{1}{2}\left[\epsilon_d(q_+) - \epsilon_d(q_-)\right] + \frac{1}{2}J_d + W_d}\right]
\end{aligned}\end{equation}





\newpage

The various processes contributing to its renormalisation can be classified into the following groups.
\subsection{Through spin-flip scattering within \(f-\) and \(d-\)layers}
We first consider scattering processes that integrate out particle (occupied) states of the high-energy subspace. One such process is
\begin{equation}\begin{aligned}
	\Delta_{1} = \frac{1}{N^2}\sum_{{\bf q} \in \text{UV}}\sum_{{\bf k} \in \text{IR}}\frac{1}{2}J S_f^+ S_d^- G_{2,f}(\epsilon_{{\bf q},f} - \epsilon_{{\bf k},f}) \frac{1}{2}J_f S_f^- c^\dagger_{{\bf q}\uparrow}c_{{\bf k}\downarrow} G_{1,f}(\epsilon_{{\bf k},f} - \epsilon_{{\bf q},f}, {\bf k}, {\bf q}) \frac{1}{2}J_fS_f^+ c^\dagger_{{\bf k}\downarrow}c_{{\bf q}\uparrow}~,
\end{aligned}\end{equation}
where \(G_{1,f}\) and \(G_{2,f}\) are propagators for the first and second intermediate states:
\begin{equation}\begin{aligned}
	G_{1,f}(E, {\bf k}, {\bf q}) &= \frac{1}{\omega - \frac{1}{2}E + \frac{1}{4}\left(J_f({\bf k}) + J_f({\bf q})\right) + \frac{1}{2}\left(W_f({\bf k}) + W_f({\bf q})\right)}~,\quad G_{2,f}(E) &= \frac{1}{\omega - \frac{1}{2}E}~.
\end{aligned}\end{equation}
Upon contracting the momentum field operators, we get
\begin{equation}\begin{aligned}
	\Delta_{1} &= S_f^+ S_d^- \frac{1}{N^2}\frac{1}{8}J J_f^2 \sum_{{\bf q} \in \text{UV}}\sum_{{\bf k} \in \text{IR}}G_{1,f}(\epsilon_{{\bf k},f} - \epsilon_{{\bf q},f}, {\bf k}, {\bf q}) G_{2,f}(\epsilon_{{\bf q},f} - \epsilon_{{\bf k},f}) n_{{\bf q}\uparrow}\left(1 - n_{{\bf k}\downarrow}\right)\\
			   &= S_f^+ S_d^- \frac{1}{8}J J_f^2 \int_\text{UV} d\epsilon_1 \int_\text{IR} d\epsilon_2 \sum_{{\bf q} \in \epsilon_1}~\sum_{{\bf k} \in \epsilon_2}G_{1,f}(\epsilon_2 - \epsilon_1, {\bf k}, {\bf q}) G_{2,f}(\epsilon_1 - \epsilon_2) n_{{\bf q}\uparrow}\left(1 - n_{{\bf k}\downarrow}\right) \\ 
			   &= S_f^+ S_d^- \frac{1}{8}J J_f^2 \int_{-D}^{-(D - \delta D)} d\epsilon_1 \int_0^D d\epsilon_2 \sum_{{\bf q} \in \epsilon_1}~\sum_{{\bf k} \in \epsilon_2}G_{1,f}(\epsilon_2 - \epsilon_1, {\bf k}, {\bf q}) G_{2,f}(\epsilon_1 - \epsilon_2)~.
\end{aligned}\end{equation}

A very similar process can be constructed by reversing the sequence of operations in the present process, while staying in the occupied sector:
\begin{equation}\begin{aligned}
	\Delta_{2} = \frac{1}{N^2}\sum_{{\bf q} \in \text{UV}}\sum_{{\bf k} \in \text{IR}}\frac{1}{2}J_fS_f^+ c^\dagger_{{\bf q}\downarrow}c_{{\bf k}\uparrow}  G_{1,f}(\epsilon_{{\bf k},f} - \epsilon_{{\bf q},f}, {\bf k}, {\bf q}) \frac{1}{2}J_f S_f^- c^\dagger_{{\bf k}\uparrow}c_{{\bf q}\downarrow} G_{2,f}(\epsilon_{{\bf q},f} - \epsilon_{{\bf k},f}) \frac{1}{2}J S_f^+ S_d^- ~.
\end{aligned}\end{equation}
This leads to an exchange of the intermediate states, leaving the product of propagators \(G_{1,f} G_{2,f}\) unchanged. Combining with the previous contribution, we get
\begin{equation}\begin{aligned}
	\Delta_{1}  + \Delta_2 &= S_f^+ S_d^- \frac{1}{8}J J_f^2 \int_\text{UV} d\epsilon_1 \int_\text{IR} d\epsilon_2 \sum_{{\bf q} \in \epsilon_1}~\sum_{{\bf k} \in \epsilon_2}G_{1,f}(\epsilon_2 - \epsilon_1, {\bf k}, {\bf q}) G_{2,f}(\epsilon_1 - \epsilon_2) \sum_\sigma n_{{\bf q}\sigma}\left(1 - n_{{\bf k}\bar\sigma}\right)\\
						   &= S_f^+ S_d^- \frac{1}{4}J J_f^2 \int_{-D}^{-(D - \delta D)} d\epsilon_1 \int_0^D d\epsilon_2 \sum_{{\bf q} \in \epsilon_1}~\sum_{{\bf k} \in \epsilon_2}G_{1,f}(\epsilon_2 - \epsilon_1, {\bf k}, {\bf q}) G_{2,f}(\epsilon_1 - \epsilon_2)~.
\end{aligned}\end{equation}

The particle-hole transformed contribution is easily obtained by switching the momentum indices in the propagators. In total, we get
\begin{equation}\begin{aligned}
	\Delta_{1}  + \Delta_2 = S_f^+ S_d^- \frac{1}{4}J J_f^2 &\left[\int_{-D}^{-(D - \delta D)} d\epsilon_1 \int_0^D d\epsilon_2 \sum_{{\bf q} \in \epsilon_1}~\sum_{{\bf k} \in \epsilon_2}G_{1,f}(\epsilon_2 - \epsilon_1, {\bf k}, {\bf q}) G_{2,f}(\epsilon_1 - \epsilon_2) \right.\\
															&+\left. \int_{D - \delta D}^{D} d\epsilon_1 \int_{-D}^0 d\epsilon_2 \sum_{{\bf q} \in \epsilon_1}~\sum_{{\bf k} \in \epsilon_2}G_{1,f}(\epsilon_1 - \epsilon_2, {\bf k}, {\bf q}) G_{2,f}(\epsilon_2 - \epsilon_1)\right]~\\
	=S_f^+ S_d^- \frac{1}{4}J J_f^2 &\int_{D - \delta D}^{D} d\epsilon_1 \int_{-D}^0 d\epsilon_2 \sum_{{\bf q} \in \epsilon_1}~\sum_{{\bf k} \in \epsilon_2}\left[G_{1,f}(\epsilon_1 - \epsilon_2, {\bf k}, {\bf q}) G_{2,f}(\epsilon_2 - \epsilon_1) + G_{1,f}(\epsilon_1 - \epsilon_2, {\bf k} + {\bf Q}_2, {\bf q} + {\bf Q}_2) G_{2,f}(\epsilon_2 - \epsilon_1)\right]~,
\end{aligned}\end{equation}
where we substituted \(\epsilon_i \to -\epsilon_i\) in the first integral and used \(\epsilon_{\bf q} = -\epsilon_{{\bf q} + {\bf Q}_2}\), with \({\bf Q}_2 = \left(\pi, \pi\right)\). Using the properties of \(J\) and \(W\), we can show that \(J({\bf k}_1) = J({\bf k}_1 + {\bf Q}_2)\). This property allows us to equate the two products within the integrand:
\begin{equation}\begin{aligned}
	\Delta_{1,f}  + \Delta_{2,f} = \frac{1}{2}S_f^+ S_d^- J J_f^2 &\int_{D - \delta D}^{D} d\epsilon_1 \int_{-D}^0 d\epsilon_2 \sum_{{\bf q} \in \epsilon_1}~\sum_{{\bf k} \in \epsilon_2}G_{1,f}(\epsilon_1 - \epsilon_2, {\bf k}, {\bf q}) G_{2,f}(\epsilon_2 - \epsilon_1) ~.
\end{aligned}\end{equation}
where the subscript \(f\) indicates that this takes into account spin-flip scattering processes within the \(f-\)layer.

Carrying out the same computation for the \(d-\)layer simply requires changing all \(f-\)quantities to \(d-\)quantities:
\begin{equation}\begin{aligned}
	\Delta_{1,d}  + \Delta_{2,d} = \frac{1}{2}S_f^+ S_d^- J J_d^2 &\int_{D - \delta D}^{D} d\epsilon_1 \int_{-D}^0 d\epsilon_2 \sum_{{\bf q} \in \epsilon_1}~\sum_{{\bf k} \in \epsilon_2}G_{1,d}(\epsilon_1 - \epsilon_2, {\bf k}, {\bf q}) G_{2,d}(\epsilon_2 - \epsilon_1) ~,
\end{aligned}\end{equation}
where 
\begin{equation}\begin{aligned}
	G_{1,d}(E, {\bf k}, {\bf q}) &= \frac{1}{\omega - \frac{1}{2}E + \frac{1}{4}\left(J_d({\bf k}) + J_d({\bf q})\right) + \frac{1}{2}\left(W_d({\bf k}) + W_d({\bf q})\right)}~,\quad G_{2,d}(E) &= \frac{1}{\omega - \frac{1}{2}E}~.
\end{aligned}\end{equation}




\end{document}
