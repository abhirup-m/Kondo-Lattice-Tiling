\documentclass[%
reprint,
superscriptaddress,
groupedaddress,
superscriptaddress,
onecolumn,
]{revtex4-2}
\usepackage[margin=1cm]{geometry}
\usepackage{graphicx,braket,bm,hyperref,amsmath,amssymb}

\begin{document}

\title{Quantum criticality in a Kondo-Mott Lattice Model}

\author{Abhirup Mukherjee}
\author{Siddhartha Lal}

\date{\today}

\maketitle
\section{Impurity Model}
\begin{equation}\begin{aligned}
	H_\text{aux}({\bf r}_d) = H^{(0)} + H_f({\bf r}_d) + H_c({\bf r}_d) + H_{fc}({\bf r}_d)~,
\end{aligned}\end{equation}
\begin{equation}\begin{aligned}
	H^{(0)} =& -t_f\sum_{\braket{i,j},\sigma}\left(f^\dagger_{i,\sigma}f_{j,\sigma} + \text{h.c.}\right) -t\sum_{\braket{i,j},\sigma}\left(c^\dagger_{i,\sigma}c_{j,\sigma} + \text{h.c.}\right)~,\\
	H_f({\bf r}_d) =& V_f\sum_{Z \in \text{NN}}\sum_{\sigma}\left(f^\dagger_{{\bf r}_d,\sigma} f_{Z,\sigma} + \text{h.c.}\right) - \frac{U_f}{2}\left(f^\dagger_{{\bf r}_d,\uparrow}f_{{\bf r}_d,\uparrow} - f^\dagger_{{\bf r}_d,\downarrow}f_{{\bf r}_d,\downarrow}\right)^2 \\
				   &+ J_f\sum_{Z \in \text{NN}}\sum_{\alpha,\beta}{\bf S}_f({\bf r}_d) \cdot {\boldsymbol{\sigma}}_{\alpha\beta}f^\dagger_{Z,\alpha}f_{Z,\beta} - \frac{W_f}{2} \sum_{Z \in \text{NN}}\left(f^\dagger_{Z,\uparrow}f_{Z,\uparrow} - f^\dagger_{Z,\downarrow}f_{Z,\downarrow} \right)^2~,\\
	H_c({\bf r}_d) =& - \frac{W}{2}\left(c^\dagger_{{\bf r}_d,\uparrow}c_{{\bf r}_d,\uparrow} - c^\dagger_{{\bf r}_d,\downarrow}c_{{\bf r}_d,\downarrow} \right)^2~,\\
	H_{fc}({\bf r}_d) =& J\sum_{\alpha,\beta}{\bf S}_f({\bf r}_d) \cdot {\boldsymbol{\sigma}}_{\alpha\beta}c^\dagger_{{\bf r}_d,\alpha}c_{{\bf r}_d,\beta} + V \left(f^\dagger_{{\bf r}_d,\sigma} c_{{\bf r}_d,\sigma} + \text{h.c.}\right)~,\\
\end{aligned}\end{equation}

\section{Tiling reconstruction}
\begin{equation}\begin{aligned}
	H_\text{tiled} =& \sum_{{\bf r}_d}H_\text{aux}({\bf r}_d) - (N-1) H^{(0)} \\
	=& \sum_{\braket{i,j},\sigma}\left[-t_f\left(f^\dagger_{i,\sigma}f_{j,\sigma} + \text{h.c.}\right) -t\left(c^\dagger_{i,\sigma}c_{j,\sigma} + \text{h.c.}\right)\right] + \tilde J \sum_{\braket{i,j}}{\bf S}_f(i) \cdot {\bf S_f}(j) + J\sum_i {\bf S}_f(i)\cdot {\bf S}_c(i) - U\sum_i \left(f^\dagger_{i,\uparrow}f_{i,\uparrow} - f^\dagger_{i,\downarrow}f_{i,\downarrow}\right)^2
\end{aligned}\end{equation}

\section{Coupling renormalisation group flows}
Off-diagonal terms:
\begin{equation}\begin{aligned}
	H_{X, f} &= \frac{1}{2}\sum_{{\bf q},{\bf k},\sigma}J_f({\bf k},{\bf q}) S_f^\sigma \left(f^\dagger_{{\bf q},-\sigma}f_{{\bf k},\sigma} + f^\dagger_{{\bf k},-\sigma}f_{{\bf q},\sigma}\right) + \frac{1}{2}\sum_{{\bf q},{\bf k},\sigma}J_f({\bf k},{\bf q}) \sigma S_f^z \left(f^\dagger_{{\bf q},\sigma}f_{{\bf k},\sigma} + f^\dagger_{{\bf k},\sigma}f_{{\bf q},\sigma}\right)~,\\
	H_{X, c} &= \frac{1}{2}J\sum_{{\bf q},{\bf k},\sigma} S_f^\sigma \left(c^\dagger_{{\bf q},-\sigma}c_{{\bf k},\sigma} + c^\dagger_{{\bf k},-\sigma}c_{{\bf q},\sigma}\right) + \frac{1}{2}J\sum_{{\bf q},{\bf k},\sigma} \sigma S_f^z \left(c^\dagger_{{\bf q},\sigma}c_{{\bf k},\sigma} + c^\dagger_{{\bf k},\sigma}c_{{\bf q},\sigma}\right)~.
\end{aligned}\end{equation}

\subsection{Intra-layer processes}
\begin{equation}\begin{aligned}
	\Delta J_f^{(j)}\left({\bf k}_1, {\bf k}_2\right) &= -\sum_{{\bf q} \in \text{PS}} \left(J_f^{(j)}\left({\bf k}_2,{\bf q}\right) J_f^{(j)}\left({\bf q},{\bf k}_1\right) + 4J_f^{(j)}\left({\bf q}, {\boldsymbol \pi} + {\bf q}\right) W_{{\boldsymbol \pi} + {\bf q}, {\bf k}_2, {\bf k}_1, {\bf q}}\right) G_f(\omega, {\bf q})~,\\
	\Delta J^{(j)} &= - \rho(\varepsilon_j)\Delta \varepsilon\cdot\left[\left(J^{(j)}\right)^2 + 4WJ^{(j)}\right]~G(\omega, {\bf q})~,
\end{aligned}\end{equation}
where \({\bf \bar q} = {\boldsymbol \pi} + {\bf q}\) is the charge conjugate partner of \({\bf q}\), and the propagators \(G_f\) and \(G\) are defined as
\begin{equation}\begin{aligned}\label{propagators}
	G_f(\omega,{\bf q}) &= \left(\omega - \frac{1}{2}|\varepsilon_f({\bf q})| + J^{(j)}_f\left({\bf q},{\bf q}\right)/4 + W_f({\bf q})/2 + \mu_f\right)^{-1} ~,\\
	G(\omega,{\bf q}) &= \frac{1}{2}\left[\left(\omega - \frac{1}{2}|\varepsilon({\bf q})| + J^{(j)}/4 + W/2 + \mu/2\right)^{-1} + \left(\omega - \frac{1}{2}|\varepsilon({\bf q})| + J^{(j)}/4 + W/2 - \mu/2\right)^{-1}\right] ~.
\end{aligned}\end{equation}


\subsection{Inter-layer processes}
Processes that start from configurations in which \({\bf q}\) is occupied:
\begin{equation}\begin{aligned}
	P_1 = \frac{1}{8}\sum_{{\bf q},{\bf k},{\bf k}_1, {\bf k}_2,\sigma,\sigma^\prime} J_f^{(j)}({\bf q},{\bf k}) S_f^\sigma f^\dagger_{{\bf q},-\sigma}f_{{\bf k},\sigma} ~ G_f(\omega, {\bf q}) ~ J^{(j)} S_f^z \sigma^\prime c^\dagger_{{\bf k}_1,\sigma^\prime}c_{{\bf k}_2,\sigma^\prime} ~ G_f(\omega, {\bf q}) ~ J_f^{(j)}({\bf q},{\bf k}) S_f^{-\sigma} f^\dagger_{{\bf k},\sigma}f_{{\bf q},-\sigma}~,
\end{aligned}\end{equation}
where the propagator \(G_f(\omega)\) for the excitations is defined in eq.~\ref{propagators}:
Using \(S^\sigma S^z = -\frac{\sigma}{2}S^\sigma\) and \(S^\sigma S^{-\sigma} = \frac{1}{2} + \sigma S^z\), we get
\begin{equation}\begin{aligned}
	P_1 &= \sum_{{\bf k}_1, {\bf k}_2,\sigma,\sigma^\prime} \frac{-\sigma^\prime}{16} \left(\frac{\sigma}{2} + S_f^z\right) c^\dagger_{{\bf k}_1,\sigma^\prime}c_{{\bf k}_2,\sigma^\prime} ~  J^{(j)} \sum_{{\bf q} \in \text{PS} ,{\bf k} \in \text{HS}}\left(J_f^{(j)}({\bf q},{\bf k})\right)^2 G_f^2(\omega, {\bf q})\\
		&= -\frac{1}{8}\sum_{{\bf k}_1, {\bf k}_2,\sigma^\prime} \sigma^\prime~S_f^z c^\dagger_{{\bf k}_1,\sigma^\prime}c_{{\bf k}_2,\sigma^\prime} ~  J^{(j)} \sum_{{\bf q} \in \text{PS} ,{\bf k} \in \text{HS}}\left(J_f^{(j)}({\bf q},{\bf k})\right)^2 G^2_f(\omega, {\bf q})~.
\end{aligned}\end{equation}

Another process can be conceived with similar starting configuration but where the loop momenta are on the \(c-\)plane:
\begin{equation}\begin{aligned}
	P_2 = \frac{1}{8}\sum_{{\bf q},{\bf k},{\bf k}_1, {\bf k}_2,\sigma,\sigma^\prime} J^{(j)} S_f^\sigma c^\dagger_{{\bf q},-\sigma}c_{{\bf k},\sigma} ~ G(\omega, {\bf q}) ~ J^{(j)}({\bf k}_1,{\bf k}_2) S_f^z \sigma^\prime f^\dagger_{{\bf k}_1,\sigma^\prime}f_{{\bf k}_2,\sigma^\prime} ~ G(\omega, {\bf q}) ~ J^{(j)} S_f^{-\sigma} c^\dagger_{{\bf k},\sigma}c_{{\bf q},-\sigma}~,
\end{aligned}\end{equation}
with the propagator defined in eq.~\ref{propagators}.

Using the same properties as above, the expression simplifies to:
\begin{equation}\begin{aligned}
	P_2 &= -\frac{1}{8}\sum_{{\bf k}_1, {\bf k}_2,\sigma^\prime} \sigma^\prime~S_f^z f^\dagger_{{\bf k}_1,\sigma^\prime}f_{{\bf k}_2,\sigma^\prime} ~  J_f^{(j)}({\bf k}_1,{\bf k}_2) \left(J^{(j)}\right)^2 \sum_{{\bf q} \in \text{HS} ,{\bf k} \in \text{PS}}G^2(\omega, {\bf q})~.
\end{aligned}\end{equation}

It turns out that the processes that start from unoccupied configurations in the state \({\bf q}\) give almost identical contributions as the present expressions, the only difference being that \({\bf q}\) is now summed over the hole sector (HS) while \({\bf k}\) is summed over the particle sector (PS). For a particle-hole symmetric system (which we have restricted ourselves to), these two summations are equal, so both the contributions are indeed identical. The total renormalisation can therefore be read off from the final expressions of \(P_1\) and \(P_2\) (and the hole sector is accounted for by doubling the renormalisation from just the particle sector):
\begin{equation}\begin{aligned}
	\Delta J_f^{(j)}({\bf k}_1,{\bf k}_2) = -\frac{1}{2}J_f^{(j)}({\bf k}_1,{\bf k}_2) \left(J^{(j)}\right)^2 \sum_{{\bf q} \in \text{PS} ,{\bf k} \in \text{HS}}G^2(\omega, {\bf q})\\
	\Delta J^{(j)} = -\frac{1}{2}J^{(j)} \sum_{{\bf q} \in \text{PS} ,{\bf k} \in \text{HS}}\left(J_f^{(j)}({\bf q},{\bf k})\right)^2G^2_f(\omega, {\bf q})\\
\end{aligned}\end{equation}

\subsection{Complete coupling RG equation}
\begin{equation}\begin{aligned}\label{rgEquation}
	\Delta J_f^{(j)}\left({\bf k}_1, {\bf k}_2\right) &= -\sum_{{\bf q} \in \text{PS}} \left[\left(J_f^{(j)}\left({\bf k}_2,{\bf q}\right) J_f^{(j)}\left({\bf q},{\bf k}_1\right) + 4J_f^{(j)}\left({\bf q}, {\bf \bar q}\right) W_{{\bf \bar q}, {\bf k}_2, {\bf k}_1, {\bf q}}\right) G_f(\omega, {\bf q}) + \frac{1}{2}J_f^{(j)}({\bf k}_1,{\bf k}_2) \left(J^{(j)}\right)^2 \sum_{{\bf k} \in \text{HS}}G^2(\omega, {\bf q})\right]~,\\
	\Delta J^{(j)} &= - \rho(\varepsilon_j)\Delta \varepsilon\cdot\left[\left(J^{(j)}\right)^2 + 4WJ^{(j)}\right] G(\omega, {\bf q}) -\frac{1}{2}J^{(j)} \sum_{{\bf q} \in \text{PS} ,{\bf k} \in \text{HS}}\left(J_f^{(j)}({\bf q},{\bf k})\right)^2G^2_f(\omega, {\bf q})~,
\end{aligned}\end{equation}

\subsection{Symmetries preserved under renormalisation}
By Fourier transforming the real-space forms of the Kondo coupling \(J_f\) and bath interaction \(W\), we get their \(k-\)space forms:
\begin{equation}\begin{aligned}\label{momentumForm}
	J_f({\bf k}, {\bf q}) &= \frac{J_f}{2}\left[\cos(k_x - q_x) + \cos(k_y - q_y)\right]~,\\
	W({\bf k}, {\bf q}, {\bf k}^\prime, {\bf q}^\prime) &= \frac{W}{2}\left[\cos(k_x - q_x + k^\prime_x - q^\prime_x) + \cos(k_y - q_y + k^\prime_y - q^\prime_y)\right]~.
\end{aligned}\end{equation}
These are of course the unrenormalised forms; the Kondo coupling \(k-\)space dependence can evolve during the RG flow. The \(k-\)space sensitive form of the Kondo coupling and conduction bath interactions are invariant under symmetry transformations in the Brillouin zone.\\

\par\noindent{\bf Translation by a nesting vector into opposite quadrant}\\
Define the reciprocal lattice vectors (RLVs) \({\bf Q}_1 = \left(\pi, \pi\right)\) and \({\bf Q}_2 = \left(\pi, -\pi\right)\). The bare Kondo coupling in eq.~\ref{momentumForm} is (anti)symmetric under translation of (one)both momentum by either of the two RLVs:
\begin{equation}\begin{aligned}
	J_f({\bf k} + {\bf Q}_i, {\bf q}) = J_f({\bf k}, {\bf q} + {\bf Q}_i) = - J_f({\bf k}, {\bf q})~;~i = 1,2~;\\
	J_f({\bf k} + {\bf Q}_i, {\bf q} + {\bf Q}_j) = J_f({\bf k}, {\bf q})~;~i=1,2~~j=1,2~.\\
\end{aligned}\end{equation}
These symmetries survive under the renormalisation group transformations. For the first transformation (under which the Kondo coupling is antisymmetric), note that each of the terms on the right hand side of the RG equation for \(J_f\) (eq.~\ref{rgEquation}) are antisymmetric as well - the third term because it's the Kondo coupling itself which automatically has the symmetry, the second term (involving \(W\)) because \(W\) is also antisymmetric under transformation of one momentum, and the first term because only one of the two \(J_f\) in the product will transform (to obtain a minus sign). This ensures that the entirety of the renormalisation transforms antisymmetrically. A very similar argument shows that the symmetry under transformation of both momenta also survives under renormalisation.\\

\par\noindent{\bf Translation into adjacent quadrant}\\
We will make use of another symmetry. Consider two momenta \({\bf k}\) and \({\bf q}\) in the first and second quadrant, with the same \(y-\)component but opposite \(x-\)components:
\begin{equation}\begin{aligned}
	{\bf k}_y = {\bf q}_y, ~ {\bf k}_x = -{\bf q}_x~.
\end{aligned}\end{equation}
We refer to \({\bf q}\) as \({\bf \bar k}\) to signal the fact the above relation between the two momenta. We first consider the bare interaction, where we have the symmetry
\begin{equation}\begin{aligned}
	J_f({\bf k}, {\bf \bar k}^\prime) = J_f({\bf \bar k}, {\bf k}^\prime),~\\
	J_f({\bf \bar k}, {\bf \bar k}^\prime) = J_f({\bf k}, {\bf k}^\prime)~.
\end{aligned}\end{equation}
We now argue that these symmetries are preserved during the RG flow. Using the properties \(W_{{\bf \bar q}, {\bf k}_2, {\bf \bar k}_1, {\bf q}} = W_{{\bf \bar q}, {\bf \bar k}_2, {\bf k}_1, {\bf q}}\) and  \(J_f^{(j)}\left({\bf k}_2,{\bf q}\right) J_f^{(j)}\left({\bf q},{\bf \bar k}_1\right) = J_f^{(j)}\left({\bf \bar k}_2,{\bf \bar q}\right) J_f^{(j)}\left({\bf \bar q},{\bf k}_1\right)\) and the fact that \({\bf \bar q}\) lies on the same isoenergy shell as \({\bf q}\) and is already part of the summation over PS in eq.~\ref{rgEquation}, we can see that \(\Delta J_f^{(j)}\left({\bf \bar k}_1, {\bf k}_2\right) = \Delta J_f^{(j)}\left({\bf k}_1, {\bf \bar k}_2\right)\). A similar line of argument shows that \(\Delta J_f^{(j)}\left({\bf \bar k}_1, {\bf \bar k}_2\right) = \Delta J_f^{(j)}\left({\bf k}_1, {\bf k}_2\right)\).


\section{Reduction to Truncated 1D Representation}
At the renormalisation group fixed point, we have a renormalised theory for the interaction of the impurity spin \(S_f\) with the \(f-\)layer Fermi surface:
\begin{equation}\begin{aligned}
	H^* = \sum_{\alpha,\beta}{\bf S}_f\cdot{\boldsymbol\sigma}_{\alpha,\beta}\sum_{{\bf k}, {\bf q}} J_f^*({\bf k},{\bf q}) f^\dagger_{{\bf k},\alpha}f_{{\bf q},\beta}~.
\end{aligned}\end{equation}
We will now obtain a more minimal representation of this interaction. 


\end{document}
